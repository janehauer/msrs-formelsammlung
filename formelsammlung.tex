\documentclass[a4paper, 11pt]{article}
\usepackage[a4paper, left=3cm, right=2cm, top=2cm]{geometry}
\usepackage[utf8]{inputenc}					% Zeichenkodierung UTF-8 falls Probleme wegen utf8 auftreten, utf8 durch utf8x ersetzen
\usepackage[ngerman]{babel}					% Deutsche Sprache und Silbentrennung
\usepackage{xcolor}
\usepackage{amsmath}						% erlaubt mathematische Formeln
\usepackage{amssymb}						% Verschiedene Symbole
\usepackage{graphicx}						% Zum Bilder einfügen benötigt
\usepackage{hyperref}						% Sprunglinks für Überschriften, Fußnoten und Weblinks
\usepackage{tabularx}
\usepackage{siunitx}                        % Zur Darstellung von SI Einheiten
\sisetup{
  locale = DE ,
  per-mode = symbol
}
\usepackage{multicol}                       % Mehrspaltig

\usepackage{parskip}                        % No Space after newline

\newcommand\mainformular[1]{\fbox{#1}}      % Standard Formel
\newcommand\legende[1]{
    \colorbox{lightblue}{%
    \begin{tabularx}{\textwidth}{llll}
     #1
    \end{tabularx}
    }
}                                           % Farbig hintelegte Legende
\definecolor{lightblue}{RGB}{204, 230, 255}

\begin{document}

\section{Messtechnik}

\subsection{Grundlagen Drehspulmesser}
\subsubsection{Windungen im Wickelraum}
\begin{minipage}{0.45\textwidth} 
\mainformular{ $A_W = N \cdot d^2$} \\
\end{minipage} 
\begin{minipage}{0.45\textwidth} 
 
\legende{ 
$A_W$ & Wickelraum & & \\
$N$ & Anzahl der Windungen &  & \\
$d^2$ & Drahtdurchmesser & \si{\square\metre} & \\
} 
\end{minipage}

\subsubsection{Elektrisches Moment} 
\begin{minipage}{0.45\textwidth} 
\mainformular{ $M_{el}= A \cdot N \cdot B \cdot I$} \\
\end{minipage} 
\begin{minipage}{0.45\textwidth} 
 
\legende{ 
$N$ & Anzahl der Windungen &  & \\
$I$ & Stromstärke & \si{\ampere} & \\
$A$ & Fläche & \si{\square\metre}  & \\
$B$ & Feldstärke & \si{\tesla} & \\
} 
\end{minipage}

\subsubsection{Mechanisches Moment} 
\begin{minipage}{0.45\textwidth} 
\mainformular{ $M_{mech}= \alpha \cdot D$} \\
\end{minipage} 
\begin{minipage}{0.45\textwidth} 
\legende{ 
$D$ & Federkonstante & $\si{\newton\metre}/90\si{\degree}$ & \\
$\alpha$ & Ausschlagwinkel & \si{\degree} & \\
} 
\end{minipage}

\subsubsection{Zeigerausschlag}
\begin{minipage}{0.45\textwidth}
\mainformular{ $\alpha = I \cdot \cfrac{A\cdot N\cdot B}{D}$}
\end{minipage}
\begin{minipage}{0.45\textwidth}
\legende{
$N$ & Anzahl der Windungen &  & \\
$I$ & Stromstärke & \si{\ampere} & \\
$A$ & Fläche & \si{\square\metre}  & \\
$D$ & Federkonstante & \si{\newton\metre} & \\
}
\end{minipage}

\subsubsection{Strommessung mit Nebenwiderstand}
\begin{minipage}{0.45\textwidth}
\mainformular{ $(I-I_M) R_N= I_M(R_M+R_V)$ }\\
$R_N = \cfrac{I_M(R_M+R_V)}{I-I_M} $
\end{minipage}
\begin{minipage}{0.45\textwidth}
\legende{
$I_M$ & Messwerkstrom & \si{\ampere} &  \\
& 1\si{\milli\ampere} oder 100\si{\micro\ampere} & & \\
$I$ & Stromstärke & \si{\ampere} & \\
$R_M$ & Spulenwiederstand (Kupfer*) & \si{\ohm}  & \\
$R_N$ &  & \si{\ohm} & \\
$R_V$ &  & \si{\ohm} & \\
}
*Temperaturkoeffizient Kupfer: $4\%/10\si{\kelvin}$
\end{minipage}


\subsubsection{Güteklasse mit Temperaturkoeffizient} 
\begin{minipage}{0.45\textwidth} 
\mainformular{ $G = \cfrac{R_M}{R_M+R_V}\cdot 4\%/10 \si{\kelvin}$ } \\
\end{minipage} 
\begin{minipage}{0.45\textwidth} 
 
\legende{ 
$G$ & Güteklasse & & \\
$R_M$ & Spulenwiederstand (Kupfer*) & \si{\ohm}  & \\
$R_N$ &  & \si{\ohm} & \\
$R_V$ &  & \si{\ohm} & \\
} 
\end{minipage}

\subsubsection{Rückwirkungsfehler Strommessung} 
\begin{minipage}{0.45\textwidth} 
\mainformular{ $F_I = \cfrac{I_M - I_0}{I_0} = -\cfrac{R_M}{R_0 + R_L + R_M}$ } \\
\end{minipage} 
\begin{minipage}{0.45\textwidth}  
\legende{ 
$F_I$ & systemischer Fehler & & \\
$I_0$ & &  \si{\ampere} & \\
$I_M$ & &  \si{\ampere} & \\
$R_0$ & &  \si{\ohm} & \\
$R_L$ & Lastwiderstand & \si{\ohm} & \\
$R_M$ & Spulenwiederstand (Kupfer*) & \si{\ohm}  & \\
} 
\end{minipage}

\subsubsection{Spannungsmesser} 
\begin{minipage}{0.45\textwidth} 
\mainformular{ $R_V = \cfrac{U}{I_M} - R_M$ } \\
\end{minipage} 
\begin{minipage}{0.45\textwidth}  
\legende{ 
$I_M$ & &  \si{\ampere} & \\
$R_M$ & Spulenwiederstand (Kupfer*) & \si{\ohm}  & \\
$R_V$ & Vorwiderstand & \si{\ohm} & \\
$U$ & Spannung & \si{\volt} & \\
} 
\end{minipage}

\subsubsection{Rückwirkungsfehler Spannungsmessung} 
\begin{minipage}{0.45\textwidth} 
\mainformular{ $F_U = \cfrac{U_M - U_0}{U_0} = -\cfrac{R_0}{R_0 + R_i}$ } \\
$U_M = \cfrac{U_0}{R_0+R_i}R_i$ \\
$R_i = R_M + R_V$
\end{minipage} 
\begin{minipage}{0.45\textwidth}  
\legende{ 
$F_U$ & systemischer Fehler & & \\
$U_0$ & &  \si{\volt} & \\
$U_M$ & &  \si{\volt} & \\
$R_0$ & &  \si{\ohm} & \\
$R_i$ & & \si{\ohm} & \\
$R_M$ & Spulenwiederstand (Kupfer*) & \si{\ohm}  & \\
$R_V$ & Vorwiderstand & \si{\ohm} & \\
} 
\end{minipage}


\subsection{Grundlagen DVN}
\subsubsection{DVN Genauigkeit Bit}
\begin{minipage}{0.45\textwidth}
\mainformular{ $B(n) = \cfrac{\log(2\cdot 10^n)}{\log(2)}$}
\end{minipage}
\begin{minipage}{0.45\textwidth}
\legende{
$n$ & Stellen der Anzeige & $\mathbb{N}$ & \\
}
\end{minipage}

\subsubsection{DVN Genauigkeit \%} 
\begin{minipage}{0.45\textwidth} 
\mainformular{ $e_r= \cfrac{1}{2 \cdot 10^n - 1}$} \\
$e_r= \cfrac{1}{2^{B(n)} - 1}$
\end{minipage} 
\begin{minipage}{0.45\textwidth} 
 
\legende{ 
$n$ & Stellen der Anzeige & $\mathbb{N}$ &  \\ 
 \\ 
} 
\end{minipage} 

\subsubsection{Anzeigen Auflösung} 

Bestimmung durch den Kehrwert der Anzeige. Beispiel für $3\frac{1}{2}$ 

\begin{minipage}{0.45\textwidth} 
\mainformular{ $0.5 \cdot 10^{-3}$} \\
\end{minipage} 
\begin{minipage}{0.45\textwidth} 
\end{minipage} 

\subsubsection{Spannung pro Digit} 
\begin{minipage}{0.45\textwidth} 
\mainformular{ $I_{Dig} = I \cdot n$} \\
\end{minipage} 
\begin{minipage}{0.45\textwidth} 
 
\legende{ 
$n$ & Kehrwert der Anzeige &  &  \\ 
$Mess_{max}$ & Max Wert Messbereich &  & \\
} 
\end{minipage} 

\subsubsection{Rückwirkungsfehler} 
Dieser ist größer als bei Analogen Messverfahren denn $R_P \ge R_M$.

\begin{minipage}{0.45\textwidth} 
\mainformular{ $F_I = \cfrac{I_M - I_0}{I_0} = -\cfrac{R_P}{R_0 + R_L + R_P}$ } \\
\end{minipage} 
\begin{minipage}{0.45\textwidth} 
 
\legende{ 
$F_I$ & systemischer Fehler & & \\
$I_0$ & &  \si{\ampere} & \\
$I_M$ & &  \si{\ampere} & \\
$R_0$ & &  \si{\ohm} & \\
$R_L$ & Lastwiderstand & \si{\ohm} & \\
$R_P$ & & \si{\ohm}  & \\
} 
\end{minipage}

\subsubsection{Rückwirkungsfehler Spannungsmessung} 
\begin{minipage}{0.45\textwidth} 
\mainformular{ $F_U = \cfrac{\cfrac{R_i R_P}{R_i + R_P}-R_P}{R_P} = -\cfrac{R_P}{R_i + R_P}$ } \\
\end{minipage} 
\begin{minipage}{0.45\textwidth} 
 
\legende{ 
$F_U$ & systemischer Fehler & & \\
$U_0$ & &  \si{\volt} & \\
$U_M$ & &  \si{\volt} & \\
$R_0$ & &  \si{\ohm} & \\
$R_i$ &  & \si{\ohm} & \\
$R_M$ & Spulenwiederstand (Kupfer*) & \si{\ohm}  & \\
$R_V$ & Vorwiderstand & \si{\ohm} & \\
} 
\end{minipage}


\section{Regelungstechnik}
\subsection{Stabilität von Regelkreisen}
Es gilt:

\begin{minipage}{0.45\textwidth}

\mainformular{ $F_G = \frac{F_o}{1 + F_o}$} \\
\mainformular{ $F_G = \frac{Z_o}{Z_o + N_o}$} \\
\mainformular{ $F_o = F_R \cdot F_S$} \\

\end{minipage}
\begin{minipage}{0.45\textwidth}

\legende{
$F_G$ & geschlossener Kreis  & & \\
$Z_o$ & Zähler offener Kreis  & & \\
$N_o$ & Nenner offener Kreis  & & \\
$F_o$ & offener Kreis  & & \\
$F_G$ & geschlossener Kreis  & & \\

}
\end{minipage}

\subsubsection{Hurwitz-Kriterium}
charakteristische Gleichung des geschl. Regelkreises:
$$ a_mp^m + a_{m-1}p^{m-1}+ ... + a_1p+ a_0 = 0$$

notwendige Bedingung: alle Koeffizienten der charakteristischen Gleichung des geschlossenen Regelkreises müssen vorhanden und positives Vorzeichen haben. \\

hinreichende Bedingung: Alle Hauptabschnitssdeterminanten $D_i$
der Hurwitzdeterminantte H müssen positiven Wert haben.\\

\begin{minipage}{0.45\textwidth}

\mainformular{ $D_2 = a_1\cdot a_2 - a_3 \cdot a_0$} \\
\end{minipage}
\begin{minipage}{0.45\textwidth}

\legende{
$D_2$ & Determinante rel. für System 3.Ord.  & & \\

}
\end{minipage}

\subsubsection{Niquist-Kriterium}
Der geschlossene Regelkreis ist stabil, wenn der kritische Punkt (-1,0) links der Ortskurve $F_o(j\omega)$ seines offenen Kreises liegt.

\begin{minipage}{0.45\textwidth}

\mainformular{ $F_o(j\omega) = \frac{K}{A(j\omega)+ j B(j\omega)}$} \\
$\omega_k \Rightarrow B(\omega) = 0$ \\
$\frac{K}{A(\omega_k)} > -1 $
\end{minipage}
\begin{minipage}{0.45\textwidth}

\legende{
$F_o(j\omega)$ & Übertragungsfkt. offenen Kreis  & & \\
& Berechnungen zum Prüfen d. Stabilität & & \\
}
\end{minipage}


\subsection{Regelgüte}
\begin{minipage}{0.45\textwidth}
$F_z(p) = \frac{x(p)}{Z(p)} = \frac{-F_s}{1+F_o} = 0$ \\
 $F_W(p) = \frac{x(p)}{w(p)} = \frac{F_o}{1+F_o} = 1$ \\

\end{minipage}
\begin{minipage}{0.45\textwidth}

\legende{
$F_z(p)$ & ideales Störverhalten  & & \\
$F_W(p)$ & ideales Führungsverhalten &  & \\
}
\end{minipage}

\subsubsection{Bleibende Regelabweichung Führungsverhalten}
\begin{minipage}{0.45\textwidth}
\mainformular{ $R_{1W} = \lim \limits_{p \rightarrow 0} \frac{1}{1+F_o(p)}$} \\
$R_{1WP} = \frac{1}{1+V_o}$ \\
 $R_{1WI} = 0$ \\

\end{minipage}
\begin{minipage}{0.45\textwidth}

\legende{
$R_{1W}$ & bleibende Regelabweichung  & & \\
& Führungsverhalten allgemein & & \\
$R_{1WP}$ & P-Regelkreis (ohne I-Glied) & & \\
$R_{1WI}$ & I-Regelkreis  & & \\

}
\end{minipage}

\subsubsection{Bleibende Regelabweichung Störverhalten}
\begin{minipage}{0.45\textwidth}
\mainformular{ $R_{1Z} = \lim \limits_{p \to 0} \frac{F_s(p)}{1+F_o(p)}$} \\
 $R_{1ZP} = \lim \limits_{p \to 0} \frac{F_s}{1+F_RF_S} = \frac{V_S}{1+V_RV_S} \approx \frac{1}{V_R}$ \\
$R_{1ZIS} = \lim \limits_{p \to 0} \frac{1}{pT_{IS}+ V_R} = \frac{1}{V_R}$ \\

\mainformular{ $R_{1ZIR} = \lim \limits_{p \to 0} \frac{pT_{IR}* V_S}{pT_{IR}+ V_S} = 0$} \\

\end{minipage}
\begin{minipage}{0.45\textwidth}

\legende{
$R_{1Z}$ & bleibende Regelabweichung  & & \\
& Störverhalten allgemein & & \\
$R_{1ZP}$ & P-Regelkreis (ohne I-Glied)  &  & \\
& für $V_RV_S >> 1$ & & \\
$R_{1ZIS}$ & I-Regelkreis, Strecke mit I-Glied  & & \\
$R_{1ZIR}$ & I-Regelkreis, Strecke ohne I-Glied  & & \\

}
\end{minipage}

\subsubsection{Geschwindigkeitsfehler}
Führungsgröße als Rampenfunktion \\
$ w(t) = a\cdot t \Rightarrow w(p) = \frac{a}{p^2}$\\

\begin{minipage}{0.45\textwidth}
\mainformular{ $R_{2} = \lim \limits_{p \to 0} p \cdot w(p) \frac{1}{1+F_o(p)}= 
\lim \limits_{p \to 0}  \frac{a}{p} \cdot \frac{1}{1+F_o(p)}$}  \\
$R_{2P} = \lim \limits_{p \to 0}  \frac{a}{p} \cdot \frac{1}{1+V_o}= \infty$\\
$ R_{2I} = \frac{aT_o}{V_o}$\\

\end{minipage}
\begin{minipage}{0.45\textwidth}

\legende{
$R_{2}$ & Geschwindigkeitsfehler  & & \\
&  allgemein & & \\
$R_{2P}$ & P-Regelkreis (ohne I-Glied)  &  & \\
$R_{2I}$ & I-Regelkreis  & & \\

}
\end{minipage}


\end{document}

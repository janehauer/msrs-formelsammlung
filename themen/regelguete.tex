\begin{minipage}{0.45\textwidth}
$F_z(p) = \frac{x(p)}{Z(p)} = \frac{-F_s}{1+F_o} = 0$ \\
 $F_W(p) = \frac{x(p)}{w(p)} = \frac{F_o}{1+F_o} = 1$ \\

\end{minipage}
\begin{minipage}{0.45\textwidth}

\legende{
$F_z(p)$ & ideales Störverhalten  & & \\
$F_W(p)$ & ideales Führungsverhalten &  & \\
}
\end{minipage}

\subsubsection{Bleibende Regelabweichung Führungsverhalten}
\begin{minipage}{0.45\textwidth}
\mainformular{ $R_{1W} = \lim \limits_{p \rightarrow 0} \frac{1}{1+F_o(p)}$} \\
$R_{1WP} = \frac{1}{1+V_o}$ \\
 $R_{1WI} = 0$ \\

\end{minipage}
\begin{minipage}{0.45\textwidth}

\legende{
$R_{1W}$ & bleibende Regelabweichung  & & \\
& Führungsverhalten allgemein & & \\
$R_{1WP}$ & P-Regelkreis (ohne I-Glied) & & \\
$R_{1WI}$ & I-Regelkreis  & & \\

}
\end{minipage}

\subsubsection{Bleibende Regelabweichung Störverhalten}
\begin{minipage}{0.45\textwidth}
\mainformular{ $R_{1Z} = \lim \limits_{p \to 0} \frac{F_s(p)}{1+F_o(p)}$} \\
 $R_{1ZP} = \lim \limits_{p \to 0} \frac{F_s}{1+F_RF_S} = \frac{V_S}{1+V_RV_S} \approx \frac{1}{V_R}$ \\
$R_{1ZIS} = \lim \limits_{p \to 0} \frac{1}{pT_{IS}+ V_R} = \frac{1}{V_R}$ \\

\mainformular{ $R_{1ZIR} = \lim \limits_{p \to 0} \frac{pT_{IR}* V_S}{pT_{IR}+ V_S} = 0$} \\

\end{minipage}
\begin{minipage}{0.45\textwidth}

\legende{
$R_{1Z}$ & bleibende Regelabweichung  & & \\
& Störverhalten allgemein & & \\
$R_{1ZP}$ & P-Regelkreis (ohne I-Glied)  &  & \\
& für $V_RV_S >> 1$ & & \\
$R_{1ZIS}$ & I-Regelkreis, Strecke mit I-Glied  & & \\
$R_{1ZIR}$ & I-Regelkreis, Strecke ohne I-Glied  & & \\

}
\end{minipage}

\subsubsection{Geschwindigkeitsfehler}
Führungsgröße als Rampenfunktion \\
$ w(t) = a\cdot t \Rightarrow w(p) = \frac{a}{p^2}$\\

\begin{minipage}{0.45\textwidth}
\mainformular{ $R_{2} = \lim \limits_{p \to 0} p \cdot w(p) \frac{1}{1+F_o(p)}=
\lim \limits_{p \to 0}  \frac{a}{p} \cdot \frac{1}{1+F_o(p)}$}  \\
$R_{2P} = \lim \limits_{p \to 0}  \frac{a}{p} \cdot \frac{1}{1+V_o}= \infty$\\
$ R_{2I} = \frac{aT_o}{V_o}$\\

\end{minipage}
\begin{minipage}{0.45\textwidth}

\legende{
$R_{2}$ & Geschwindigkeitsfehler  & & \\
&  allgemein & & \\
$R_{2P}$ & P-Regelkreis (ohne I-Glied)  &  & \\
$R_{2I}$ & I-Regelkreis  & & \\

}
\end{minipage}
\newpage
\subsubsection{Integralkriterien}
Um die Regelgüte zu bestimmen wird aus der Sprungantwort berechnet\\


\begin{minipage}{0.45\textwidth}
\mainformular{ $R_{2} = \lim \limits_{p \to 0} p \cdot w(p) \frac{1}{1+F_o(p)}=
\lim \limits_{p \to 0}  \frac{a}{p} \cdot \frac{1}{1+F_o(p)}$}  \\

$IE = \int_{0}^{\infty}(x_{\infty})-x(t))dt = \int_{0}^{\infty} \Delta xdt $\\
$IAE = \int_{0}^{\infty}  | x_{\infty}-x(t) | dt $\\
$ ISE = \int_{0}^{\infty} (x_{\infty}-x(t))^2dt = \int_{0}^{\infty}(e(t))^2 dt  $\\

$ ISE_1 = \frac{c_0^2}{2d_0d_1}$\\
$ ISE_2 = \frac{c_1^2d_0+c_0^2d_2}{2d_0d_1d_2}$ \\
$ ISE_3 = \frac{c_2^2d_0d_1+c_0^2d_2d_3+(c_1^2-2c_0c_2)d_0d_3}{2d_0d_3(d_1d_2-d_0d_3)}$\\

$ ITAE =  \int_{0}^{\infty}t |\Delta x|dt $\\
$ ITSE =  \int_{0}^{\infty}t (\Delta x)^2dt $\\

\end{minipage}
\begin{minipage}{0.45\textwidth}

\legende{
$IE$ & lineare Regelfläche   & & \\
$IAE$ & betragslineare Regelfläche   &  & \\
$ISE$ & quadratische Regelfläche  & & \\
$ITAE$ & zeitbewertete   &  & \\
$ITSE$ & zeitbewertete  & & \\
}
\end{minipage}

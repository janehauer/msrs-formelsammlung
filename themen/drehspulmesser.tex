\subsubsection{Windungen im Wickelraum}
\begin{minipage}{0.45\textwidth} 
\mainformular{ $A_W = N \cdot d^2$} \\
\end{minipage} 
\begin{minipage}{0.45\textwidth} 
 
\legende{ 
$A_W$ & Wickelraum & & \\
$N$ & Anzahl der Windungen &  & \\
$d^2$ & Drahtdurchmesser & \si{\square\metre} & \\
} 
\end{minipage}

\subsubsection{Elektrisches Moment} 
\begin{minipage}{0.45\textwidth} 
\mainformular{ $M_{el}= A \cdot N \cdot B \cdot I$} \\
\end{minipage} 
\begin{minipage}{0.45\textwidth} 
 
\legende{ 
$N$ & Anzahl der Windungen &  & \\
$I$ & Stromstärke & \si{\ampere} & \\
$A$ & Fläche & \si{\square\metre}  & \\
$B$ & Feldstärke & \si{\tesla} & \\
} 
\end{minipage}

\subsubsection{Mechanisches Moment} 
\begin{minipage}{0.45\textwidth} 
\mainformular{ $M_{mech}= \alpha \cdot D$} \\
\end{minipage} 
\begin{minipage}{0.45\textwidth} 
\legende{ 
$D$ & Federkonstante & $\si{\newton\metre}/90\si{\degree}$ & \\
$\alpha$ & Ausschlagwinkel & \si{\degree} & \\
} 
\end{minipage}

\subsubsection{Zeigerausschlag}
\begin{minipage}{0.45\textwidth}
\mainformular{ $\alpha = I \cdot \cfrac{A\cdot N\cdot B}{D}$}
\end{minipage}
\begin{minipage}{0.45\textwidth}
\legende{
$N$ & Anzahl der Windungen &  & \\
$I$ & Stromstärke & \si{\ampere} & \\
$A$ & Fläche & \si{\square\metre}  & \\
$D$ & Federkonstante & \si{\newton\metre} & \\
}
\end{minipage}

\subsubsection{Strommessung mit Nebenwiderstand}
\begin{minipage}{0.45\textwidth}
\mainformular{ $(I-I_M) R_N= I_M(R_M+R_V)$ }\\
$R_N = \cfrac{I_M(R_M+R_V)}{I-I_M} $
\end{minipage}
\begin{minipage}{0.45\textwidth}
\legende{
$I_M$ & Messwerkstrom & \si{\ampere} &  \\
& 1\si{\milli\ampere} oder 100\si{\micro\ampere} & & \\
$I$ & Stromstärke & \si{\ampere} & \\
$R_M$ & Spulenwiederstand (Kupfer*) & \si{\ohm}  & \\
$R_N$ &  & \si{\ohm} & \\
$R_V$ &  & \si{\ohm} & \\
}
*Temperaturkoeffizient Kupfer: $4\%/10\si{\kelvin}$
\end{minipage}


\subsubsection{Güteklasse mit Temperaturkoeffizient} 
\begin{minipage}{0.45\textwidth} 
\mainformular{ $G = \cfrac{R_M}{R_M+R_V}\cdot 4\%/10 \si{\kelvin}$ } \\
\end{minipage} 
\begin{minipage}{0.45\textwidth} 
 
\legende{ 
$G$ & Güteklasse & & \\
$R_M$ & Spulenwiederstand (Kupfer*) & \si{\ohm}  & \\
$R_N$ &  & \si{\ohm} & \\
$R_V$ &  & \si{\ohm} & \\
} 
\end{minipage}

\subsubsection{Rückwirkungsfehler Strommessung} 
\begin{minipage}{0.45\textwidth} 
\mainformular{ $F_I = \cfrac{I_M - I_0}{I_0} = -\cfrac{R_M}{R_0 + R_L + R_M}$ } \\
\end{minipage} 
\begin{minipage}{0.45\textwidth}  
\legende{ 
$F_I$ & systemischer Fehler & & \\
$I_0$ & &  \si{\ampere} & \\
$I_M$ & &  \si{\ampere} & \\
$R_0$ & &  \si{\ohm} & \\
$R_L$ & Lastwiderstand & \si{\ohm} & \\
$R_M$ & Spulenwiederstand (Kupfer*) & \si{\ohm}  & \\
} 
\end{minipage}

\subsubsection{Spannungsmesser} 
\begin{minipage}{0.45\textwidth} 
\mainformular{ $R_V = \cfrac{U}{I_M} - R_M$ } \\
\end{minipage} 
\begin{minipage}{0.45\textwidth}  
\legende{ 
$I_M$ & &  \si{\ampere} & \\
$R_M$ & Spulenwiederstand (Kupfer*) & \si{\ohm}  & \\
$R_V$ & Vorwiderstand & \si{\ohm} & \\
$U$ & Spannung & \si{\volt} & \\
} 
\end{minipage}

\subsubsection{Rückwirkungsfehler Spannungsmessung} 
\begin{minipage}{0.45\textwidth} 
\mainformular{ $F_U = \cfrac{U_M - U_0}{U_0} = -\cfrac{R_0}{R_0 + R_i}$ } \\
$U_M = \cfrac{U_0}{R_0+R_i}R_i$ \\
$R_i = R_M + R_V$
\end{minipage} 
\begin{minipage}{0.45\textwidth}  
\legende{ 
$F_U$ & systemischer Fehler & & \\
$U_0$ & &  \si{\volt} & \\
$U_M$ & &  \si{\volt} & \\
$R_0$ & &  \si{\ohm} & \\
$R_i$ & & \si{\ohm} & \\
$R_M$ & Spulenwiederstand (Kupfer*) & \si{\ohm}  & \\
$R_V$ & Vorwiderstand & \si{\ohm} & \\
} 
\end{minipage}
